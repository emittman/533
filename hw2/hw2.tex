\documentclass[12pt]{article}\usepackage[]{graphicx}\usepackage[]{color}
%% maxwidth is the original width if it is less than linewidth
%% otherwise use linewidth (to make sure the graphics do not exceed the margin)
\makeatletter
\def\maxwidth{ %
  \ifdim\Gin@nat@width>\linewidth
    \linewidth
  \else
    \Gin@nat@width
  \fi
}
\makeatother

\definecolor{fgcolor}{rgb}{0.345, 0.345, 0.345}
\newcommand{\hlnum}[1]{\textcolor[rgb]{0.686,0.059,0.569}{#1}}%
\newcommand{\hlstr}[1]{\textcolor[rgb]{0.192,0.494,0.8}{#1}}%
\newcommand{\hlcom}[1]{\textcolor[rgb]{0.678,0.584,0.686}{\textit{#1}}}%
\newcommand{\hlopt}[1]{\textcolor[rgb]{0,0,0}{#1}}%
\newcommand{\hlstd}[1]{\textcolor[rgb]{0.345,0.345,0.345}{#1}}%
\newcommand{\hlkwa}[1]{\textcolor[rgb]{0.161,0.373,0.58}{\textbf{#1}}}%
\newcommand{\hlkwb}[1]{\textcolor[rgb]{0.69,0.353,0.396}{#1}}%
\newcommand{\hlkwc}[1]{\textcolor[rgb]{0.333,0.667,0.333}{#1}}%
\newcommand{\hlkwd}[1]{\textcolor[rgb]{0.737,0.353,0.396}{\textbf{#1}}}%

\usepackage{framed}
\makeatletter
\newenvironment{kframe}{%
 \def\at@end@of@kframe{}%
 \ifinner\ifhmode%
  \def\at@end@of@kframe{\end{minipage}}%
  \begin{minipage}{\columnwidth}%
 \fi\fi%
 \def\FrameCommand##1{\hskip\@totalleftmargin \hskip-\fboxsep
 \colorbox{shadecolor}{##1}\hskip-\fboxsep
     % There is no \\@totalrightmargin, so:
     \hskip-\linewidth \hskip-\@totalleftmargin \hskip\columnwidth}%
 \MakeFramed {\advance\hsize-\width
   \@totalleftmargin\z@ \linewidth\hsize
   \@setminipage}}%
 {\par\unskip\endMakeFramed%
 \at@end@of@kframe}
\makeatother

\definecolor{shadecolor}{rgb}{.97, .97, .97}
\definecolor{messagecolor}{rgb}{0, 0, 0}
\definecolor{warningcolor}{rgb}{1, 0, 1}
\definecolor{errorcolor}{rgb}{1, 0, 0}
\newenvironment{knitrout}{}{} % an empty environment to be redefined in TeX

\usepackage{alltt}
\usepackage[margin=1in]{geometry}
\usepackage{amsmath}
\usepackage{graphicx}
\author{Eric Mittman}
\title{Assignment 2}
\IfFileExists{upquote.sty}{\usepackage{upquote}}{}
\begin{document}

\maketitle
\begin{enumerate}
\item[2.2.] It is possible for a continuous cdf to be constant over some intervals of time.\\
(a) Give an example of a physical situation that would result in a cdf F(t) that is constant over some values of t.\\
(b) Sketch such a cdf and its corresponding pdf.\\
(c) For your example, explain why the convention for defining quantiles given in Section 2.1.2 is sensible. Are there alternative definitions that would also be suitable?\\
\begin{enumerate}
  \item
  A container of sand containing buried object of interest is being poured out. When half of the sand has been poured, the pouring stops to allow search for the object. The ``time to failure" is the time at which the object is emitted from the container.
  
  \item
\begin{knitrout}
\definecolor{shadecolor}{rgb}{0.969, 0.969, 0.969}\color{fgcolor}
\includegraphics[width=.8\textwidth]{figure/1b-1} 

\end{knitrout}
  \item Since half the sand has been poured, defining $t_{.5}$ to be the 15 minutes, the first time at which half of the sand has been poured is sensible because that is the probability that the object is in the poured sand at 15 minutes.
  
  An alternative definition would be that the $p^{th}$ quantile is the largest value of $t$ such that $F(t) \le p$. This would lead to $t_{.5} = 30$ minutes.
\end{enumerate}


\item[2.6.] Consider a random variable with cdf $F(t) = t/2$, $0 < t \le 2$. Do the following:\\
  (a) Derive expressions for the corresponding pdf and hazard functions.\\
(b) Use the results of part (a) to verify the relationship given in (2.2).\\
(c) Sketch (or use the computer to draw) the cdf and pdf functions.\\
(d) Sketch (or use the computer to draw) the hazard function. Give a clear intuitive reason for the behavior of $h(t)$ as $t  \rightarrow 2$.\\
Hint : By the time $t = 2$, all units in the population must have failed.
\begin{enumerate}
  \item \[f(t) = 1/2, \quad t\in (0,2)\]
  \[h(t) = 1/(2-t), \quad t \in (0,2)\]
  \item 
  \[1-\exp\left[ -\int_{0}^{t}\frac{1}{2-x}dx\right] = 1-\exp\left[\int_{2}^{2-t}\frac{1}{u}du\right] = 1-\exp\left[ \log\left(\frac{2-t}{2}\right) \right] = t/2=F(t)\]
  \item
\begin{knitrout}
\definecolor{shadecolor}{rgb}{0.969, 0.969, 0.969}\color{fgcolor}
\includegraphics[width=.8\textwidth]{figure/1c-1} 

\end{knitrout}
  \item 
\begin{knitrout}
\definecolor{shadecolor}{rgb}{0.969, 0.969, 0.969}\color{fgcolor}\begin{figure}[ht]
\includegraphics[width=.8\textwidth]{figure/1d-1} \caption[The hazard function goes to infinity at ]{The hazard function goes to infinity at $x=2$ because as the cdf reaches 1 at 2. As the remaining time to fail decreases to zero, then the conditional rate of failure must increase without bound.}\label{fig:1d}
\end{figure}


\end{knitrout}
 \item
 \[F(t)=p \iff t=2p \implies t_{.4}=.8\]
 \item
 Since the pdf is constant over the support of $T$,
 \[Pr(.1<T\le.2) = P(.8<T\le.9) = 1/2 \cdot .1 = .05\]
 \item
 \[Pr(.1<T\le.2|T>.1) = \frac{.05}{1-F(.1)}=\frac{.05}{.95} = \frac{1}{19}\]
  \[Pr(.8<T\le.9|T>.8) = \frac{.05}{1-F(.8)}=\frac{.05}{.6} = \frac{1}{12}\]
  These are the same as what we would get by the approximation in (2.1):
  \[h(.1)\times.1 = 1/19,\quad h(.8)\times.1 = 1/12\]
 \item
 The approximation is exact in this case because the density is constant. If the pdf was decreasing on $(t,t+\triangle t)$, then (2.1) would be biased high. It would be biased low if the pdf was increasing on the interval.
\end{enumerate}

\item[2.8]
\begin{enumerate}
  \item
  \[f(t) = e^{-t},\quad h(t) = \frac{e^{-t}}{e^{-t}}=1\]
  \item
\begin{knitrout}
\definecolor{shadecolor}{rgb}{0.969, 0.969, 0.969}\color{fgcolor}
\includegraphics[width=.8\textwidth]{figure/3b-1} 

\end{knitrout}
  \item
  \item
\end{enumerate}
\end{enumerate}
\end{document}
