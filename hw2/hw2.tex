\documentclass[12pt]{article}\usepackage[]{graphicx}\usepackage[]{color}
%% maxwidth is the original width if it is less than linewidth
%% otherwise use linewidth (to make sure the graphics do not exceed the margin)
\makeatletter
\def\maxwidth{ %
  \ifdim\Gin@nat@width>\linewidth
    \linewidth
  \else
    \Gin@nat@width
  \fi
}
\makeatother

\definecolor{fgcolor}{rgb}{0.345, 0.345, 0.345}
\newcommand{\hlnum}[1]{\textcolor[rgb]{0.686,0.059,0.569}{#1}}%
\newcommand{\hlstr}[1]{\textcolor[rgb]{0.192,0.494,0.8}{#1}}%
\newcommand{\hlcom}[1]{\textcolor[rgb]{0.678,0.584,0.686}{\textit{#1}}}%
\newcommand{\hlopt}[1]{\textcolor[rgb]{0,0,0}{#1}}%
\newcommand{\hlstd}[1]{\textcolor[rgb]{0.345,0.345,0.345}{#1}}%
\newcommand{\hlkwa}[1]{\textcolor[rgb]{0.161,0.373,0.58}{\textbf{#1}}}%
\newcommand{\hlkwb}[1]{\textcolor[rgb]{0.69,0.353,0.396}{#1}}%
\newcommand{\hlkwc}[1]{\textcolor[rgb]{0.333,0.667,0.333}{#1}}%
\newcommand{\hlkwd}[1]{\textcolor[rgb]{0.737,0.353,0.396}{\textbf{#1}}}%

\usepackage{framed}
\makeatletter
\newenvironment{kframe}{%
 \def\at@end@of@kframe{}%
 \ifinner\ifhmode%
  \def\at@end@of@kframe{\end{minipage}}%
  \begin{minipage}{\columnwidth}%
 \fi\fi%
 \def\FrameCommand##1{\hskip\@totalleftmargin \hskip-\fboxsep
 \colorbox{shadecolor}{##1}\hskip-\fboxsep
     % There is no \\@totalrightmargin, so:
     \hskip-\linewidth \hskip-\@totalleftmargin \hskip\columnwidth}%
 \MakeFramed {\advance\hsize-\width
   \@totalleftmargin\z@ \linewidth\hsize
   \@setminipage}}%
 {\par\unskip\endMakeFramed%
 \at@end@of@kframe}
\makeatother

\definecolor{shadecolor}{rgb}{.97, .97, .97}
\definecolor{messagecolor}{rgb}{0, 0, 0}
\definecolor{warningcolor}{rgb}{1, 0, 1}
\definecolor{errorcolor}{rgb}{1, 0, 0}
\newenvironment{knitrout}{}{} % an empty environment to be redefined in TeX

\usepackage{alltt}
\usepackage[margin=1in]{geometry}
\usepackage{amsmath}
\usepackage{graphicx}
\author{Eric Mittman}
\title{Assignment 2}
\IfFileExists{upquote.sty}{\usepackage{upquote}}{}
\begin{document}
\maketitle
\begin{enumerate}
\item[2.2.] It is possible for a continuous cdf to be constant over some intervals of time.
(a) Give an example of a physical situation that would result in a cdf F(t) that is constant over some values of t.\\
(b) Sketch such a cdf and its corresponding pdf.\\
(c) For your example, explain why the convention for defining quantiles given in Section 2.1.2 is sensible. Are there alternative definitions that would also be suitable?\\
\begin{enumerate}
  \item
  A item is known to be located along a path of known length. Suppose this length has a unit measure. Then the probability a search of length $t$ beginning at one end of the path will find the object is $F(t) = \begin{cases}t & t\in (0,1)\\ 1 & t>1\end{cases}$.  
  \item
  \item
\end{enumerate}


\item[2.6.] Consider a random variable with cdf $F(t) = t/2$, $0 < t \le 2$. Do the following:\\
  (a) Derive expressions for the corresponding pdf and hazard functions.\\
(b) Use the results of part (a) to verify the relationship given in (2.2).\\
(c) Sketch (or use the computer to draw) the cdf and pdf functions.\\
(d) Sketch (or use the computer to draw) the hazard function. Give a clear intuitive reason for the behavior of $h(t)$ as $t  \rightarrow 2$.\\
Hint : By the time $t = 2$, all units in the population must have failed.
\begin{enumerate}
  \item \[f(t) = 1/2, \quad t\in (0,2)\]
  \[h(t) = 1/t, \quad t \in (0,2)\]
  \item 
  \item 
  \item 
  
\end{enumerate}
\end{enumerate}
\end{document}
