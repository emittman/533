\documentclass{scrartcl}\usepackage[]{graphicx}\usepackage[]{color}
%% maxwidth is the original width if it is less than linewidth
%% otherwise use linewidth (to make sure the graphics do not exceed the margin)
\makeatletter
\def\maxwidth{ %
  \ifdim\Gin@nat@width>\linewidth
    \linewidth
  \else
    \Gin@nat@width
  \fi
}
\makeatother

\definecolor{fgcolor}{rgb}{0.345, 0.345, 0.345}
\newcommand{\hlnum}[1]{\textcolor[rgb]{0.686,0.059,0.569}{#1}}%
\newcommand{\hlstr}[1]{\textcolor[rgb]{0.192,0.494,0.8}{#1}}%
\newcommand{\hlcom}[1]{\textcolor[rgb]{0.678,0.584,0.686}{\textit{#1}}}%
\newcommand{\hlopt}[1]{\textcolor[rgb]{0,0,0}{#1}}%
\newcommand{\hlstd}[1]{\textcolor[rgb]{0.345,0.345,0.345}{#1}}%
\newcommand{\hlkwa}[1]{\textcolor[rgb]{0.161,0.373,0.58}{\textbf{#1}}}%
\newcommand{\hlkwb}[1]{\textcolor[rgb]{0.69,0.353,0.396}{#1}}%
\newcommand{\hlkwc}[1]{\textcolor[rgb]{0.333,0.667,0.333}{#1}}%
\newcommand{\hlkwd}[1]{\textcolor[rgb]{0.737,0.353,0.396}{\textbf{#1}}}%

\usepackage{framed}
\makeatletter
\newenvironment{kframe}{%
 \def\at@end@of@kframe{}%
 \ifinner\ifhmode%
  \def\at@end@of@kframe{\end{minipage}}%
  \begin{minipage}{\columnwidth}%
 \fi\fi%
 \def\FrameCommand##1{\hskip\@totalleftmargin \hskip-\fboxsep
 \colorbox{shadecolor}{##1}\hskip-\fboxsep
     % There is no \\@totalrightmargin, so:
     \hskip-\linewidth \hskip-\@totalleftmargin \hskip\columnwidth}%
 \MakeFramed {\advance\hsize-\width
   \@totalleftmargin\z@ \linewidth\hsize
   \@setminipage}}%
 {\par\unskip\endMakeFramed%
 \at@end@of@kframe}
\makeatother

\definecolor{shadecolor}{rgb}{.97, .97, .97}
\definecolor{messagecolor}{rgb}{0, 0, 0}
\definecolor{warningcolor}{rgb}{1, 0, 1}
\definecolor{errorcolor}{rgb}{1, 0, 0}
\newenvironment{knitrout}{}{} % an empty environment to be redefined in TeX

\usepackage{alltt}
\usepackage{amsmath}
\author{Eric Mittman}
\title{Assignment 12, Stat 533}
\IfFileExists{upquote.sty}{\usepackage{upquote}}{}
\begin{document}
\maketitle
\begin{enumerate}
\item[9.1]
\begin{align}
p(\theta|y) =& \frac{p(y|\theta)p(\theta)}{\int p(y|\theta) p(\theta) d\theta}\\
\implies p(\theta|y) =& \frac{p(y|\theta)}{\int p(y|\theta) d\theta} \frac{\int p(y|\theta)d\theta}{\int p(y|\theta) p(\theta)d\theta}\\
\implies R_1(\theta) =& R_2(\theta)\frac{\int p(y|\theta)d\theta}{\int p(y|\theta) p(\theta)d\theta}\\
\implies \frac{R_1(\theta)}{R_2(\theta)} =& \frac{\int p(y|\theta)d\theta}{\int p(y|\theta) p(\theta)d\theta}
\end{align}
This is in terms of an ``ordinary" relative likelihood $R_2$ and a ``modified" relative likelihood that incorporates a prior. We can note that the ratio will tend to 1 as the information from the data ``overwhelms" the prior.

\item[9.3]
Since the likelihood will become increasingly concentrated as we collect more data, the posterior will also, which is not what we would expect if the posterior were reflective of a static population distribution.

\item[C9.15]

I used rstan in R to fit the model, running 4 chains for 5000 iterations each and discarding the first 2500. rstan's algorithm is purported to converge quickly and produce samples with very low autocorrelation.

The results from the LUNIF, LUNIF model:
\begin{knitrout}
\definecolor{shadecolor}{rgb}{0.969, 0.969, 0.969}\color{fgcolor}\begin{kframe}
\begin{alltt}
\hlkwd{print}\hlstd{(s)}
\end{alltt}
\begin{verbatim}
## Inference for Stan model: 43d296dd25e66fc9cd44089f1b6cea2e.
## 4 chains, each with iter=5000; warmup=2500; thin=1; 
## post-warmup draws per chain=2500, total post-warmup draws=10000.
## 
##              mean se_mean        sd    2.5%      25%      50%      75%
## log_tp       8.66    0.02      0.66    7.74     8.18     8.52     9.02
## log_beta     0.55    0.01      0.33   -0.10     0.33     0.57     0.79
## eta      50506.54 2723.08 106143.37 4750.95 10050.25 18057.49 41772.04
## beta         1.84    0.02      0.61    0.90     1.39     1.76     2.20
## lp__       -77.90    0.03      1.14  -80.93   -78.36   -77.54   -77.07
##              97.5% n_eff Rhat
## log_tp       10.30  1441 1.01
## log_beta      1.18  1432 1.01
## eta      342226.42  1519 1.01
## beta          3.25  1429 1.01
## lp__        -76.78  1707 1.00
## 
## Samples were drawn using NUTS(diag_e) at Thu Apr 14 23:56:52 2016.
## For each parameter, n_eff is a crude measure of effective sample size,
## and Rhat is the potential scale reduction factor on split chains (at 
## convergence, Rhat=1).
\end{verbatim}
\end{kframe}
\end{knitrout}

The results from the LUNIF, LN model:
\begin{knitrout}
\definecolor{shadecolor}{rgb}{0.969, 0.969, 0.969}\color{fgcolor}\begin{kframe}
\begin{alltt}
\hlkwd{print}\hlstd{(s2)}
\end{alltt}
\begin{verbatim}
## Inference for Stan model: 8f6061bacff33d4af26a9961a7742f0b.
## 4 chains, each with iter=5000; warmup=2500; thin=1; 
## post-warmup draws per chain=2500, total post-warmup draws=10000.
## 
##              mean se_mean      sd    2.5%     25%      50%      75%
## log_tp       8.30    0.01    0.28    7.84    8.11     8.27     8.46
## log_beta     0.73    0.00    0.12    0.49    0.65     0.73     0.81
## eta      13001.27  120.30 5771.47 6356.17 9249.06 11652.33 15092.14
## beta         2.09    0.00    0.26    1.63    1.91     2.07     2.25
## lp__       -77.62    0.02    1.06  -80.43  -78.03   -77.29   -76.86
##             97.5% n_eff Rhat
## log_tp       8.93  2289    1
## log_beta     0.97  2781    1
## eta      27994.09  2302    1
## beta         2.64  2768    1
## lp__       -76.60  2746    1
## 
## Samples were drawn using NUTS(diag_e) at Thu Apr 14 23:57:35 2016.
## For each parameter, n_eff is a crude measure of effective sample size,
## and Rhat is the potential scale reduction factor on split chains (at 
## convergence, Rhat=1).
\end{verbatim}
\end{kframe}
\end{knitrout}

\begin{figure}[h!]
\includegraphics[height=.4\textheight]{lunif}
\includegraphics[height=.4\textheight]{lnorm}
\caption{80\% credible bands for distribution of failure times and point estimate for the same. The top figure is the diffuse prior and the bottom figure is somewhat informative for $\beta$. The informative prior controls how fast the uncertainty increases when extrapolating.}
\end{figure}

\item[9.15]
\end{enumerate}

\end{document}
