\documentclass[12pt]{article}\usepackage[]{graphicx}\usepackage[]{color}
%% maxwidth is the original width if it is less than linewidth
%% otherwise use linewidth (to make sure the graphics do not exceed the margin)
\makeatletter
\def\maxwidth{ %
  \ifdim\Gin@nat@width>\linewidth
    \linewidth
  \else
    \Gin@nat@width
  \fi
}
\makeatother

\definecolor{fgcolor}{rgb}{0.345, 0.345, 0.345}
\newcommand{\hlnum}[1]{\textcolor[rgb]{0.686,0.059,0.569}{#1}}%
\newcommand{\hlstr}[1]{\textcolor[rgb]{0.192,0.494,0.8}{#1}}%
\newcommand{\hlcom}[1]{\textcolor[rgb]{0.678,0.584,0.686}{\textit{#1}}}%
\newcommand{\hlopt}[1]{\textcolor[rgb]{0,0,0}{#1}}%
\newcommand{\hlstd}[1]{\textcolor[rgb]{0.345,0.345,0.345}{#1}}%
\newcommand{\hlkwa}[1]{\textcolor[rgb]{0.161,0.373,0.58}{\textbf{#1}}}%
\newcommand{\hlkwb}[1]{\textcolor[rgb]{0.69,0.353,0.396}{#1}}%
\newcommand{\hlkwc}[1]{\textcolor[rgb]{0.333,0.667,0.333}{#1}}%
\newcommand{\hlkwd}[1]{\textcolor[rgb]{0.737,0.353,0.396}{\textbf{#1}}}%

\usepackage{framed}
\makeatletter
\newenvironment{kframe}{%
 \def\at@end@of@kframe{}%
 \ifinner\ifhmode%
  \def\at@end@of@kframe{\end{minipage}}%
  \begin{minipage}{\columnwidth}%
 \fi\fi%
 \def\FrameCommand##1{\hskip\@totalleftmargin \hskip-\fboxsep
 \colorbox{shadecolor}{##1}\hskip-\fboxsep
     % There is no \\@totalrightmargin, so:
     \hskip-\linewidth \hskip-\@totalleftmargin \hskip\columnwidth}%
 \MakeFramed {\advance\hsize-\width
   \@totalleftmargin\z@ \linewidth\hsize
   \@setminipage}}%
 {\par\unskip\endMakeFramed%
 \at@end@of@kframe}
\makeatother

\definecolor{shadecolor}{rgb}{.97, .97, .97}
\definecolor{messagecolor}{rgb}{0, 0, 0}
\definecolor{warningcolor}{rgb}{1, 0, 1}
\definecolor{errorcolor}{rgb}{1, 0, 0}
\newenvironment{knitrout}{}{} % an empty environment to be redefined in TeX

\usepackage{alltt}
\usepackage[margin=1in]{geometry}
\usepackage{amsmath}
\usepackage{graphicx}
\usepackage{verbatim}
\author{Eric Mittman}
\title{Assignment 3}
\IfFileExists{upquote.sty}{\usepackage{upquote}}{}
\begin{document}

  \maketitle
\begin{enumerate}
  \item[3.4]
% latex table generated in R 3.2.3 by xtable 1.8-0 package
% Wed Feb 03 22:29:51 2016
\begin{table}[ht]
\centering
\begin{tabular}{rrrrrr}
  \hline
time.lower & time.upper & Fhat & SE\_Fhat & Lower & Upper \\ 
  \hline
  0 &  33 & 0.00 & 0.00 & 0.00 & 0.00 \\ 
   33 &  46 & 0.08 & 0.08 & 0.02 & 0.34 \\ 
   46 &  50 & 0.17 & 0.11 & 0.05 & 0.42 \\ 
   50 &  59 & 0.25 & 0.12 & 0.10 & 0.50 \\ 
   59 &  62 & 0.33 & 0.14 & 0.15 & 0.58 \\ 
   62 &  71 & 0.42 & 0.14 & 0.21 & 0.65 \\ 
   71 &  74 & 0.50 & 0.14 & 0.28 & 0.72 \\ 
   74 &  75 & 0.58 & 0.14 & 0.35 & 0.79 \\ 
   75 &  78 & 0.67 & 0.14 & 0.42 & 0.85 \\ 
   78 &  80 & 0.83 & 0.11 & 0.58 & 0.95 \\ 
   \hline
\end{tabular}
\caption{Summary of link failure times (in 1000s of cycles) including estimated probabilities with corresponding standard errors and 90\% confidence intervals.} 
\end{table}

  
  \begin{enumerate}
    \item
    See Table 1.
    \item
    See Table 1. Each interval is constructed by a method that covers the truth at least 95\% of the time over many realizations of data generated by the assumed model.
    \item
    They are the same since there are no censored observations before the 3rd failure.
    \item
    The observer may not have been present at all times, so there could have been interval censoring. Another possibility is that the data were simply rounded. This is not likely to have a large impact on the analysis since most failures occured at different times and there are a relatively large number of possible failure times before right censoring occurred.
    \item
    We might expect less variability in the failure times if only one or two heats were used.
    \item
    It seems that the state of the test stand may change through the testing process in a way so as to increase or decrease the failure times. If we notice more failures at the beginning or end of the process, this possibility ought to be considered.
  \end{enumerate}
  
\item[C3.24]
\begin{enumerate}
  \item
  \[L(\pi) \propto \pi_1^4 \pi_2^5 \pi_3^2 \pi_4^{95} (\pi_2+\pi_3+\pi_4)^{99} (\pi_3+\pi_4)^{95} \]
  \item
  \[L(p) \propto p_1^4 (1-p_1)^{296} p_2^5 (1-p_2)^{192} p_3^2 (1-p_3)^{95}\]
\end{enumerate}
\item[4.3]
\[\Phi_{sev}(z) = p \iff 1-\exp[-\exp(z)] = p \iff z=\log[-\log(1-p)]\]
\[\implies \Phi_{sev}^{-1}(p) = \log[-\log(1-p)]\]
\item[4.7]
The first and second derivatives of $h(t;\beta,\eta)$ are:
\[h'(t) = \frac{\beta(\beta-1)}{\eta^{\beta}}t^{\beta-2},\quad h''(t) =\frac{\beta(\beta-1)(\beta-2)}{\eta^{\beta}}t^{\beta-3}\]
\begin{enumerate}
  \item If $0<\beta<1$, then $1-\beta<0$, so $h'(t)<0$ for $t>0$. So $h(t)$ is decreasing.
  \item If $1<\beta<2$, then $\beta-1>0$ but $\beta-2<0$, so $h'(t)>0$ and $h''(t)<0$ for all $t>0$, so $h(t)$ is concave increasing.
  \item If $\beta>2$, then $h'(t)>0$ and $h''(t)>0$, so $h(t)$ is convex increasing.
\end{enumerate}

\item[3.12]
\begin{enumerate}
  \item
  We can make inference about the distribution of failure times for these detectors under conditions that are more severe than normal operating conditions. In particular, this data will be somewhat informative about the left tail of the distribution and less informative about the rest. That is because 3/4 of the failure times are right censored. On the other hand, the intervals for the interval censored observations are fairly small, so the seven failures are observed with reasonable precision.
  
  \item See Figure 1.
  \begin{figure}
    \includegraphics{np_detectors.pdf}
    
  \end{figure}
  \item See Table 2.
% latex table generated in R 3.2.3 by xtable 1.8-0 package
% Wed Feb 03 22:29:51 2016
\begin{table}[ht]
\centering
\begin{tabular}{rrrrrrr}
  \hline
time & Fhat & SE\_Fhat & L\_pw & U\_pw & L\_simul & U\_simul \\ 
  \hline
2500 & 0.04 & 0.04 & 0.01 & 0.21 & 0.00 & 0.49 \\ 
  3000 & 0.07 & 0.05 & 0.02 & 0.24 & 0.01 & 0.49 \\ 
  3500 & 0.14 & 0.07 & 0.05 & 0.32 & 0.03 & 0.52 \\ 
  3600 & 0.18 & 0.07 & 0.08 & 0.36 & 0.04 & 0.54 \\ 
  3700 & 0.21 & 0.08 & 0.10 & 0.40 & 0.05 & 0.57 \\ 
  3800 & 0.25 & 0.08 & 0.12 & 0.44 & 0.07 & 0.60 \\ 
   \hline
\end{tabular}
\caption{Summary of detector failure times (in hours) including estimated probabilities, standard errors, pointwise and simultaneous 95\% confidence intervals.} 
\end{table}

  \item
  See Table 2 and Fig 1.
  
  \item
  See Table 2.
  
  \item
  The simultaneous intervals are wider to acheive 95\% confidence in covering all relevant quantities simultaneously. This could be useful in identifying departures from our model assumptions. If we fit a parametric model to these data, we would reasonably expect that the curve (conditional expectation) should stay within the simultaneous confidence band.

\end{enumerate}

\item[4.8]
Let $Y = \log(T), T\sim WEIB(\eta,\beta)$.
\[P(Y<y) = P(T<\exp(y)) \]
\[= 1 -\exp\left\{ - \frac{\exp(\beta y)}{\eta^\beta}\right\} \]
\[=1 -\exp\left\{ - \frac{\exp(\beta y)}{\exp(\beta \log \eta)}\right\}\]
\[=1 -\exp\left\{ - \exp(\beta (y -  \log \eta)\right\}\]
\[= \Phi_{SEV}(\log(\eta),1/\beta)]


\item[C4.22]
$V = -\log(1-U)\sim EXP(1)$. To see this, note that
\[P(V<v) = P(-\log(1-U)<v) \]
\[= P(U<1-\exp(-v))\]
\[= 1-\exp(-v),\]
which is the distribution function of an exponential(1) random variable.
\end{enumerate}
\end{document}
